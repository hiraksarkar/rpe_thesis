\chapter{Discussion and Future Work}
\label{sec:conclusion}

In this report we broadly outlined the concept of \eq in the context of RNA-seq experiments, and have shown two useful applications: \rapclust and \quark. 
\rapclust implements a fast and accurate methodology for the data-driven clustering of \denovo transcriptome assemblies. By making explicit the connection of the method of \corset~\citep{corset} and the fragment equivalence classes that have recently been proven useful in the development of accurate and fast RNA-seq quantification tools~\cite{sailfish, salmon, kallisto}, we have demonstrated how state-of-the-art transcript clustering results can be obtained much more quickly than is possible with existing tools. Furthermore, when working directly in the compact and efficient representation of fragment equivalence classes, this clustering requires only a marginal computational and storage cost beyond what is already required for \qm-based transcript-level quantification. 




\section{Possible extension of \rapclust}

There are many interesting directions for future work on this problem. Here, we mention only the most obvious. First, we believe that the quality of the resulting clusters could be improved through a data-driven selection of the appropriate cutoff parameters.  Currently, we have directly adopted the parameters suggested in~\citep{corset}, which are selected partly independent of the underlying data.  Preliminary experiments have suggested that one might be able to obtain substantially better clusters by selecting the transcript count cutoff, an edge-weight cutoff, and the log fold-change likelihood cutoff in a manner that is more data-adaptive.  However, automatically selecting these cutoffs in a general, yet rigorous fashion is a topic for future work.  Another potential improvement on the current methodology would be to adopt a more robust log fold-change test, that may be more accurate in separating contigs that do not originate from the same gene. \sailfish is capable of producing not only transcript-level abundances, but also estimates of the variance of each predicted abundance via posterior Gibbs sampling or bootstraps.  This variance information can be incorporated into the estimates of log fold-change differences to allow for increased precision in separating potential paralogs. While the existing method works well in the completely \denovo context (i.e. even when the genomes or transcriptomes of closely related organisms may not be available), integrating homology information, when available, has the potential to improve the clustering results (and provide meaningful biological annotations for the clusters). The best way to integrate this information is an exciting direction for future work. Finally, we believe that sequence-level comparison and analysis of the resulting clusters of contigs could reveal important information about the nature of the transcripts present in the samples.  For example, one could imagine ``reverse-engineering'' the splicing patterns present in the transcripts occurring in the same cluster.  This would allow one to build a virtual gene model, even in the completely \denovo context, which could then be used downstream, such as for differential splicing analyses. 


The concept of {\it equivalence classes} is generic and can be extended in multiple ways, such as identification of novel transcripts, discovering splice sites, improving quality of \denovo assembly etc. In case of annotated reference, \eq represents possible isoforms, whereas in \denovo world the meaning of \eq is closely related to sequences that potentially share a strong sequence similarity, given the read data. This connection can be used to identify novel transcripts while performing the mapping and quantification process. It would be interesting to explore the correlation between change of structure of \eq in the framework of control-treatment experiment. The \denovo assembly process can be substantially improved by collapsing \eq from multiple samples and conditions. 

%We use an expectation-maximization (EM) algorithm to manipulate edge weights in the graph based on the relationship within contigs from the annotated species and the sequence similarity between contigs from the two organisms (explained in more detail in the Online Methods section). Annotations are mapped onto the graph using BLAST between the sequences. The EM algorithm then assigns annotations to unlabeled contigs using the graph-based semi-supervised learning algorithm, Modified Adsorption (MAD, implemented in publically available software - add an explanation of why we use this method). . In order to further improve contig annotation, edge weights are changed (explained in detail in Online Methods) and the label propogation algorithm re-run. This is repeated until the number of edges in the graph converges. We show that this method significantly increases the number of annotated contigs as compared to a simple BLAST (add results showing numbers, add a sentence about result) and also improves the quality of clustering contigs. 
