%% ----------------------------------------------------------------
%% Thesis.tex -- MAIN FILE (the one that you compile with LaTeX)
%% ---------------------------------------------------------------- 

% Set up the document
\documentclass[a4paper, 11pt, oneside]{Thesis}  % Use the "Thesis" style, based on the ECS Thesis style by Steve Gunn
\graphicspath{Figures/}  % Location of the graphics files (set up for graphics to be in PDF format)

% Include any extra LaTeX packages required
\usepackage[round, comma]{natbib}  % Use the "Natbib" style for the references in the Bibliography
\usepackage{verbatim}  % Needed for the "comment" environment to make LaTeX comments
\usepackage{vector}  % Allows "\bvec{}" and "\buvec{}" for "blackboard" style bold vectors in maths
\usepackage{comment}
\usepackage{xcolor}
\usepackage{soul}
%%%% commands
\newcommand{\note}[1]{\textcolor{red}{#1}}
%%%%


\bibliographystyle{plainnat}

\hypersetup{urlcolor=blue, colorlinks=true}  % Colours hyperlinks in blue, but this can be distracting if there are many links.

%% ----------------------------------------------------------------
\begin{document}
\frontmatter      % Begin Roman style (i, ii, iii, iv...) page numbering

% Set up the Title Page
\title  {Thesis Title}
\authors  {\texorpdfstring
            {\href{http://www.hiraksarkar.com}{Hirak Sarkar}}
            {Hirak Sarkar}
            }
%\addresses  {\groupname\\\deptname\\\univname}  % Do not change this here, instead these must be set in the "Thesis.cls" file, please look through it instead
\date       {\today}
\subject    {}
\keywords   {}

\maketitle
%% ----------------------------------------------------------------

\setstretch{1.3}  % It is better to have smaller font and larger line spacing than the other way round

% Define the page headers using the FancyHdr package and set up for one-sided printing
%\fancyhead{}  % Clears all page headers and footers
%\rhead{\thepage}  % Sets the right side header to show the page number
%\lhead{}  % Clears the left side page header
\pagestyle{fancy}
%\renewcommand{\subsubsectionmark}[1]{\markright{\thesubsubsection\ #1}}
\fancyhf{}
%\rhead{\fancyplain{}{$<$Name of the report$>$}} % predefined ()
%\lhead{\fancyplain{}{\rightmark }}
%\fancyhead[L]{\emph{\rightmark}}
%\fancyhead[R]{\thepage}
  % Finally, use the "fancy" page style to implement the FancyHdr headers

%% ----------------------------------------------------------------
% Declaration Page required for the Thesis, your institution may give you a different text to place here
\begin{comment}
\Declaration{

\addtocontents{toc}{\vspace{1em}}  % Add a gap in the Contents, for aesthetics

I, AUTHOR NAME, declare that this thesis titled, `THESIS TITLE' and the work presented in it are my own. I confirm that:

\begin{itemize} 
\item[\tiny{$\blacksquare$}] This work was done wholly or mainly while in candidature for a research degree at this University.
 
\item[\tiny{$\blacksquare$}] Where any part of this thesis has previously been submitted for a degree or any other qualification at this University or any other institution, this has been clearly stated.
 
\item[\tiny{$\blacksquare$}] Where I have consulted the published work of others, this is always clearly attributed.
 
\item[\tiny{$\blacksquare$}] Where I have quoted from the work of others, the source is always given. With the exception of such quotations, this thesis is entirely my own work.
 
\item[\tiny{$\blacksquare$}] I have acknowledged all main sources of help.
 
\item[\tiny{$\blacksquare$}] Where the thesis is based on work done by myself jointly with others, I have made clear exactly what was done by others and what I have contributed myself.
\\
\end{itemize}
 
 
Signed:\\
\rule[1em]{25em}{0.5pt}  % This prints a line for the signature
 
Date:\\
\rule[1em]{25em}{0.5pt}  % This prints a line to write the date
}
\clearpage  % Declaration ended, now start a new page
\end{comment}
%% ----------------------------------------------------------------
% The "Funny Quote Page"
\pagestyle{empty}  % No headers or footers for the following pages

\null\vfill
% Now comes the "Funny Quote", written in italics
\textit{``Nature uses only the longest threads to weave her patterns, so that each small piece of her fabric reveals the organization of the entire tapestry.''}

\begin{flushright}
— Richard P. Feynman
\end{flushright}

\vfill\vfill\vfill\vfill\vfill\vfill\null
\clearpage  % Funny Quote page ended, start a new page
%% ----------------------------------------------------------------

% The Abstract Page
\addtotoc{Abstract}  % Add the "Abstract" page entry to the Contents
\abstract{
\addtocontents{toc}{\vspace{1em}}  % Add a gap in the Contents, for aesthetics

Recent technological advances in high-throughput sequencing have posed a new challenge of designing robust and efficient algorithms for downstream analysis. This rapid growth has also widened the scope for application of more sophisticated statistical techniques to discover biological insights. In this report, we have focused on analysis of biological sequencing data derived from RNA-seq, an experimental protocol for assaying the entire transcriptome ( i.e., the collection of expressed transcripts) of a given sample. RNA-seq is tremendously useful for a wide variety of tasks ranging from differential expression analysis to the discovery of previously unannotated genes and transcripts. In this work, we used the initial mapping results from \qm, a state-of-the-art mapping algorithm, to construct a more compact representation, termed as {\it equivalence classes} of the RNA-seq experiment. We have shown the usefulness of such representation via two different applications: \rapclust and \quark..

Our first tool, \rapclust is designed to assist in the analysis of \denovo transcriptomes. \rapclust introduces a novel and highly efficient method for clustering contigs from \denovo transcriptome assemblies, exploiting the multi-mapping sequence fragments as evidence of contigs that may derive from the same underlying gene. Specifically, we cast the problem of clustering contigs as the problem of clustering a sparse graph. Compared to other state-of-the-art tools, \rapclust is substantially faster.  Moreover, \rapclust produces clusters that with a closer correspondence to the true genes of origin than other tools.  The methodology of \rapclust also yields accurate estimates of expression for the resulting clusters (i.e., genes). Our second tool, \quark, uses the concept of {\it equivalence classes} to compress RNA-seq read data. \quark provides state-of-the-art compression rates, and introduces the concept of ``semi-reference-based" compression, in which a reference is assumed at the compressor, but not required at the decompressor.


%Recent technological advances in high throughput sequencing has posed a new challenge of designing robust and efficient algorithms for analysis. This rapid growth has also widen the scope for application of more sophisticated statistical techniques to discover biological insights. In this report we have focused on analysis of biological sequencing data derived from RNA-seq. RNA-seq is an experimental protocol for assaying the entire transcriptome ( i.e. the collection of expressed transcripts) of a given sample. RNA-seq is tremendously useful for various range of tasks from differential expression analysis to the discovery of previously unannotated genes and transcripts. We use the initial mapping results from \qm: a state of the art mapping algorithm, to construct a more compact representation, termed as {\it equivalence classes} of the RNA-seq experiment. We have shown the usefulness such representation via two different applications.

%Our first tool, \rapclust is designed to assist in the analysis of \denovo transcriptomes. \rapclust introduces a novel and highly-efficient method for clustering contigs from de novo transcriptome assemblies, exploiting the multi-mapping sequence fragments as evidence of contigs that may derive from the same underlying gene. To be more specific we cast the problem of clustering contigs as the problem of clustering a sparse graph. Compared to other state of the art tools, \rapclust is tremendously faster.  Moreover, \rapclust produces clusters that more closely correspond to the true genes of origin than other tools.  The methodology of \rapclust also yields accurate estimates of expression for the resulting clusters (i.e. genes). Our second tool \quark uses the concept of {\it equivalence classes} to compress RNA-seq read data. \quark provides state-of-the-art compression rates, and introduces the concept of ``semi-reference-based" compression, in which a reference is assumed at the compressor, but not required at the decompressor.

%My research focuses on the design of robust and efficient algorithms for analyzing biological data, with the goal of obtaining biological insights from typically large datasets. The major concentration of my study has been the analysis of biological sequencing data — specifically, RNA-seq. RNA-seq is an experimental protocol for assaying the entire transcriptome ( i.e. the collection of expressed transcripts) of a given sample. RNA-seq is tremendously useful for various range of tasks from differential expression analysis to the discovery of previously unannotated genes and transcripts. 

%A crucial application of RNA-seq is de-novo transcriptome analysis, in which we wish to study the transcriptomes of organisms for which no genomic assembly is available.  We have developed a tool, RapClust, to assist in the analysis of de novo transcriptomes. RapClust introduces a novel and highly-efficient method for clustering contigs from de novo transcriptome assemblies, exploiting the multi-mapping sequence fragments as evidence of contigs that may derive from the same underlying gene. To be more specific we cast the problem of clustering contigs as the problem of clustering a sparse graph. Compared to other state of the art tools, RapClust is tremendously faster.  Moreover, RapClust produces clusters that more closely correspond to the true genes of origin than other tools.  The methodology of RapClust also yields accurate estimates of expression for the resulting clusters (i.e. genes). 

%I will also discuss another of our tools, Quark (yet to release), which adopts a novel methodology  to compress RNA-seq read data. Quark provides state-of-the-art compression rates, and introduces the concept of “semi-reference-based” compression, in which a reference is assumed at the compressor, but not required at the decompressor.  Quark generates a very compact representation of the read sequences, that can be efficiently used for downstream analysis.


}

\clearpage  % Abstract ended, start a new page
%% ----------------------------------------------------------------

\setstretch{1.3}  % Reset the line-spacing to 1.3 for body text (if it has changed)

% The Acknowledgements page, for thanking everyone

\acknowledgements{
\addtocontents{toc}{\vspace{1em}}  % Add a gap in the Contents, for aesthetics

I would like to thank my advisor Prof. Rob Patro for his patience and continuous support throughout the project. My fellow students Avi, Laraib, Nitish, Komal and Mohsin for patiently listening and commenting on the progress of the project in every group meeting.
}
\clearpage  % End of the Acknowledgements
%% ----------------------------------------------------------------

\pagestyle{fancy}  %The page style headers have been "empty" all this time, now use the "fancy" headers as defined before to bring them back


%% ----------------------------------------------------------------
\lhead{\emph{Contents}}  % Set the left side page header to "Contents"
\tableofcontents  % Write out the Table of Contents

%% ----------------------------------------------------------------
\lhead{\emph{List of Figures}}  % Set the left side page header to "List if Figures"
\listoffigures  % Write out the List of Figures

%% ----------------------------------------------------------------
\lhead{\emph{List of Tables}}  % Set the left side page header to "List of Tables"
\listoftables  % Write out the List of Tables

%% ----------------------------------------------------------------
\setstretch{1.5}  % Set the line spacing to 1.5, this makes the following tables easier to read
\clearpage  % Start a new page
%\lhead{\emph{Abbreviations}}  % Set the left side page header to "Abbreviations"
%\listofsymbols{ll}  % Include a list of Abbreviations (a table of two columns)
%{
% \textbf{Acronym} & \textbf{W}hat (it) \textbf{S}tands \textbf{F}or \\
%\textbf{LAH} & \textbf{L}ist \textbf{A}bbreviations \textbf{H}ere \\

%}

%% ----------------------------------------------------------------
%\clearpage  % Start a new page
%\lhead{\emph{Physical Constants}}  % Set the left side page header to "Physical Constants"
%\listofconstants{lrcl}  % Include a list of Physical Constants (a four column table)
%{
% Constant Name & Symbol & = & Constant Value (with units) \\
%Speed of Light & $c$ & $=$ & $2.997\ 924\ 58\times10^{8}\ \mbox{ms}^{-\mbox{s}}$ (exact)\\}

%% ----------------------------------------------------------------
% \clearpage  %Start a new page
% \lhead{\emph{Symbols}}  % Set the left side page header to "Symbols"
% \listofnomenclature{lll}  % Include a list of Symbols (a three column table)
% {
% % symbol & name & unit \\
% $a$ & distance & m \\
% $P$ & power & W (Js$^{-1}$) \\
% & & \\ % Gap to separate the Roman symbols from the Greek
% $\omega$ & angular frequency & rads$^{-1}$ \\
% }
% %% ----------------------------------------------------------------
% % End of the pre-able, contents and lists of things
% % Begin the Dedication page

% \setstretch{1.3}  % Return the line spacing back to 1.3

% \pagestyle{empty}  % Page style needs to be empty for this page
% \dedicatory{For/Dedicated to/To my\ldots}

\addtocontents{toc}{\vspace{2em}}  % Add a gap in the Contents, for aesthetics


%% ----------------------------------------------------------------
\mainmatter	  % Begin normal, numeric (1,2,3...) page numbering
\pagestyle{fancy}  % Return the page headers back to the "fancy" style

% Include the chapters of the thesis, as separate files
% Just uncomment the lines as you write the chapters



%\lhead{\emph{\thesection}} 
\renewcommand{\subsectionmark}[1]{%
  \ifsubsectioninheader
    \def\subsectiontitle{: #1}%
  \else
    \def\subsectiontitle{}%
  \fi}
\newif\ifsubsectioninheader
\def\subsectiontitle{}
\fancyhead[L]{\nouppercase{\rightmark\ifsubsectioninheader\subsectiontitle\fi}}
\fancyhead[R]{\thepage}

\chapter{Introduction}
%highlight how biological analysis is driven by 
The advent of high-throughput sequencing has revolutionized the analysis of biological data. The enormous scale of technological improvement has doubled the capacity of DNA sequencing each year \citep{reuter2015high}. Be it the study of cellular processes, response of cellular pathways to drugs or transcriptional regulation, many different types of biological analysis are now highly dependant on an established pipeline of creating the experimental set up, choosing a suitable sequencing technology (sometimes designing one if needed) and sequencing a series of samples. The volume of data and speed of sequencing have widened the scope of rigorous statistical study and led to the application of sophisticated techniques from data science, e.g. as described in \citep{schatz2015biological}, in order to extract biological insights from vast volumes of sequencing data. High throughput sequencing technology also enables scientists to design protocols to serve particular applications, for example RNA-seq \citep{mortazavi2008mapping} and Ribo-seq \citep{gerashchenko2012genome} are designed to measure the level of mRNA and the rate of translation, respectively. The frequency of protein binding is revealed by ChiP-seq \citep{zhang2008model} technology, and nascent RNA transcription rate is measured by GRO-seq and PRO-seq \citep{core2008nascent} assays. Furthermore, recent single-cell technologies have enabled the probing of transcript expression profiles at the level of individual cells, which reveals the heterogeneity of different types of cells (i.e. different tissue types etc.). 

In this report we will mostly focus on RNA-seq data. Given the availability and low cost of production, RNA-seq has become one of the standard ways to measure mRNA abundance in a cell \citep{butte2000discovering}. It has wide application including \denovo transcriptome assembly, estimating isoform expression, and identifying novel transcripts and measuring differential expression under various conditions, like response to changing stimuli~\citep{Stubben2014} and disease states~\citep{diseaseDGE}.  Given the vast quantity of data often gathered to address such questions, studying these problems also requires computationally efficient methods capable of scaling with the experimental size. A number of algorithms (\citet{langmead2009ultrafast}, \citet{tophat}, \citet{mortazavi2008mapping}) have been designed to analyse the sequencing data from RNA-seq experiments. \citet{mortazavi2008mapping} have developed a statistical framework for abundance estimation from RNA-seq experiment. \citet{langmead2009ultrafast}, \citet{tophat} have  designed general purpose alignment tools for detail downstream analysis like splice detection, isoform identification etc.

%\note{\textbf{I think a bit more detail is needed here.  What challgenges did these algorithms solve?  Why did you choose these as representative? etc.}}. 

The widely-used RNA-seq data analysis pipeline, consists of three steps, first the single-end or the paired-end reads from the experiment are mapped or aligned to the reference transcriptome/genome. In case the reference transcriptome/genome is not present, either a \denovo or reference-guided assembly step is required to get an estimate of the contigs (longer contiguous region of sequences developped from raw reads) present in the sample. In the second step, abundance estimation is performed. Often, this is done by counting the reads that overlap an annotated or assembled transcript or gene, but mounting evidence has demonstrated that resolving the multi-mapping ambiguities by performing inference in an appropriate probabilistic model \citep{pachter2011models}, yields considerably more accurate abundances. After these two steps, the resulting estimates, i.e. the quantification results are used to find out the expression levels of the genes.  As described above, the variables of interest in the experimental design are used in conjunction with the quantification results to perform differential expression analysis. This allows one to study how the abundance of genes or transcripts changes with experimental factors; for example, to study how certain genes are down-regulated or up-regulated in response to a particular drug. 

It is important to note that RNA-seq experiments generate ``stable'' \citep{Chen2015} fragments from spliced transcripts, which makes the alignment step inherently difficult. A gene can express multiple isoforms, even in a single experiment. Isoforms from the same gene often share exons, which enables the generation of read sequences that, when mapped to transcriptome would naturally align with multiple transcripts at the same time. In this thesis we would propose a novel way to represent the data from a sequencing experiment by encoding the mapping information in terms of {\it equivalence classes}. Furthermore we would show two application where this representation can take aid to design robust and accurate algorithms. 

The first application aims to refine the \denovo transcriptome assemblies, resulting from different the state-of-the-art assemblers. Specifically, we design a method to cluster \denovo assembled contigs into putative genes based on shared, expressed sequence.  This enables ``gene-level'' analysis of \denovo transcriptome assemblies, mitigating the effect of assembly errors on quantification and differential analysis.  Despite the advanced algorithms used in assemblers, the reconstructed transcriptome often contains several errors including hybrid assembly where there are contigs from several gene families are put together, chimeric contigs (the spurious assembly of different underlying transcripts into a a single contig), spurious insertions and local mis-assemblies \citep{transrate}. Though improved computational methods can reduce the prevalence of such errors, the data itself is often insufficient to guarantee deterministic recovery of all expressed transcripts.  The fractured and incomplete nature of such \denovo assemblies can confound downstream analysis. We have devised a method that exploits the above mentioned equivalence classes to pose and solve a graph clustering problem, which allows us to cluster even fractured transcriptome assemblies into clusters representing putative genes. We implemented this concept as a tool \rapclust, which is explained in Chapter 2.

The second application we explore the idea of equivalence classes to derive a succinct representation of raw read sequences. From studying the mapping pattern of RNA-seq data, we have observed that a substantial fraction of reads map to a limited number of transcripts which are highly expressed in that sample. Even when multimapping is prevalent, reads tend to multimap between transcripts in consistent and repeated patterns, which can be exploited for the purposes of efficient compression. Transcripts are often covered by multi-mapping reads in a series of discontinuous region of highly abundant sequences, characterized as \islands. We design a semi-reference-based compression tool where \islands, which represent the relevant extracts from the reference, becomes a part of the encoding and later used to decode the compressed reads.
Our tool \quark, using \rapmap to map raw RNA-seq reads to the reference transcriptome. Chapter 3 is devoted for detail explanation of \quark. 

% more about equivalence classes




%For example,~\citet{corset} argue that the large number of contigs that often result from \denovo transcriptome assembly can greatly reduce the statistical power of differential expression analysis.      


\section{Alignment and Mapping}
Ambiguous mapping has made the problem of quantification extremely challenging for RNA-seq data. \citet{salzman2011statistical} and \citet{pachter2011models} described useful models that can be used for quantification under multi-mapping. \citet{sailfish} discovered the fact that a costly alignment is not really required for the purpose of quantification, related ideas that bypassed traditional notions of alignment were later used by \citet{kallisto} and \citet{salmon}. On the other hand \citet{rapmap},  proved that a stand alone exclusive mapper can be useful for multiple applications as demonstrated in Chapter 2 and Chapter 3. What all the aforementioned work share is the concept of putting similar reads together. Before we discuss the evolution of putting similar reads together in equivalence classes, let us formally define the mathematical framework of the sequence mapping/alignment paradigm. 

\subsection{Mathematical framework for nucleotide sequence matching}
Given a set of characters from an alphabet $\Sigma$, of size $|\Sigma|$, and two non-empty sequences $x,y \in \Sigma^+$, an alignment can be defined as a function $f : x \rightarrow \Sigma \cup \{-\}$, where, $``-"$ is a character to denote a gap has been introduced. Often there is a cost metric $d$ associated with the function $f$, defined as follows, 
\begin{align}
    d(x_i,y_j) &= 0,\text{ if } f(x_i) = y_j \\
    &= \text{mismatch penalty},\text{ if } f(x_i) \neq y_j \\
    &= \text{gap penalty},\text{ if } f(x_i) = ``-" 
\end{align}



The goal of most alignment algorithms \citep{Li2010} is to find alignments between the query and reference sequence to minimize the penalty of alignment, or to report if no alignment below a given penalty threshold exists. 

From the early days of bioinformatics, considerable effort \citep{Smith1981},\citep{Langmead2010},\citep{Li2008},\citep{Li2009}  has been put to develop efficient, fast and accurate sequence aligners.  

For mapping, as mentioned above, the mismatch or gaps are not considered as part of matching function. It is shown by \citet{salmon}, \citet{kallisto} and \citet{rapmap} that for certain downstream analyses, it is adequate to find out the location where read maps and the target sequence rather than full alignment information. These approaches often define mapping on the basis of a threshold. It is to be noted, although \citet{sailfish} did not map the reads explicitly, it introduced the concept eliding the traditional read alignment step which, in turn, inspired the batch of mapper dependent tools that followed it.

\subsection{Mapping in framework of RNA-seq}
Although at conceptual level RNA-seq mapping is not very different from alignment described above, but it would be worthwhile to define if formally as, these definitions are used through out the rest of the report.

Given a set of $M$ transcripts (or transcriptome) $T = \{t_1,t_2,\ldots,t_M\}$, where $t_i$ represents the nucleotide sequence for transcript $i$, likewise  $R = \{r_1,r_2,\ldots,r_N\}$ denotes a set of read sequences (might be paired end or single end), we can define the task of mapping as finding a function $\mathcal{Q} : {T,R} \rightarrow \mathcal{P}$, where $\mathcal{P}$ is a set of tuples. $\mathcal{P} = \{(r_i,t_j,p_k) : r_i \in R, t_i \in T, p_k \in \mathbb{N}\}$. The presence of tuple $(r_i,t_j,p_k)$ represents the fact that read $r_i$ is mapped to transcript $t_j$ at position $p_k$. An additional relevant information for reads with stranded protocol (sense or anti-sense) is the orientation of mapping, which often added as a flag with the mapping results
The concept of equivalence classes, described in this report, are derived from $\mathcal{P}$.  

\section{Equivalence class as a concept}

The concept of grouping similar reads together is not new. It has been mentioned several times in the literature either for improving multiple alignments or for constructing a consensus. Although it is hard to provide a full chronological account for the exact mention of such a concept in the context of computational biology, we would try to cover some important papers that implicitly made use of such a concept. 

\citet{pop2004} proposed the concept of grouping highly overlapping reads in the context of genome assembly. It has been shown that this kind of grouping greatly improve the performance and efficiency of iterative optimization algorithms such as EM \citep{sailfish}. \citet{salzman2011statistical} proposed factorization on likelihood function to speed up the quantification process. This speed up is realized by collapsing same splice junctions and exons into equivalence classes.
\citet{Salmela2011} considered the idea of grouping reads together on the basis of k-mer matches. \citet{isoem} have used similar equivalence classes over fragments and shown significant reduce in memory usage, increasing the speed of inference algorithm. Equivalence classes were defined on the pair of fragments that align to the same set of transcripts and whose compatibility weights with respect to the set of transcripts are proportional.  

% \note{Here is some text I wrote for the salmon manuscript describing the history of equivalence classes in RNA-seq explicitly.  Obviously, you shouldn't take any of this verbatim, but feel free to use it for inspiration below if you want to add anything
% Collapsing fragments into equivalence classes is a
% well-established idea in the transcript quantification literature,
% and numerous different notions of equivalence classes have
% been previously introduced, and shown to greatly reduce the time
% required to perform iterative optimization such as that described
% in XXX. For example, Salzman et al. (2011) first introduced the notion of factorizing the likelihood function to speed up inference by collapsing fragments that align to the same exons or exon junctions (as determined by a provided annotation) into equivalence classes. Simlarly, Nicolae et al. (2011) used equivalence classes over fragments to reduce memory usage and speed upinference --- they define as equivalent any pair of fragments that align to the same set of transcripts and whose compatibility weights (i.e. conditional probabilities) with respect to those transcripts are proportional.  Patro et al.(2014) define equivalence classes over k-mers, treating as equivalent any k-mers that appear in the same set of transcripts at the same frequency, and use this factorization of the likelihood function to speed up optimization.  Bray et al. (2015) define equivalence classes over fragments, and define as equivalent any fragments that pseuoalign to the same set of transcripts --- this is similar to the notion adopted by Nicolae et al., except that no restriction is placed on the proportionality of compatibilityweights (since these are not computed).}




 As mentioned before \citep{sailfish} elaborated the idea to implement a k-mer based grouping of the reads. Recent methods \citep{kallisto},\citep{salmon},\citep{rapmap} have implemented it more explicitly. 

Given this back ground on progression of development we can now formally define the idea of grouping reads and equivalence classes in the context of the current report.  

\subsection{Definition of equivalence classes}
\label{subsec:gen_equiv_classes}
We define an equivalence relation over reads, based on the set of transcripts to which they map.  The set of reads related under this definition constitutes a read equivalence class. Let $\mathcal{M}\left(r_i\right)$ be the set of transcripts to which read $r_i$ maps, and let $\mathcal{M}\left(r_j\right)$ be the set of transcripts to which read $r_j$ maps.  We say that $r_i \sim r_j$ if and only if $\mathcal{M}\left(r_i\right) = \mathcal{M}\left(r_j\right)$. Consequently, a read equivalence class is a set of reads such that, for every pair $r_i$ and $r_j$ in the class, $r_i \sim r_j$. An equivalence class can be uniquely labeled based on the set of transcripts to which the reads contained in this class map.  We define the label of equivalence class $\eqclass{r_i} = \{ r_j \in R \mid r_j \sim r_i\}$ as $\eqlabel{\eqclass{r_i}}$.  It is important to remember that, though the label consists of transcript names, the equivalence relation itself is defined over sequenced fragments and not transcripts. Finally, in addition to a label, we denote the count of each equivalence class $C_i$ by $\eqcount{C_i}$; this is simply the number of equivalent fragments in $C_i$.


Another interesting way to look at function $\mathcal{M}$ is to consider it's output on a particular set of fragments as a binary matrix. As shown in (1.4) a binary matrix can represent the read to transcript mapping function. After rearranging the rows we obtain $\mathcal{M'}$. Three equivalence classes can be extracted from here, $\{t_1,t_2,t_4\}, \{t_1\}$ and $\{t_1,t_4\}$. 

\begin{align}
  \mathcal{M}=\kbordermatrix{%
      & t_1 & t_2 & t_3 & t_4 \\
    r_1 & 1 & 1 & 0 & 1 \\
    r_2 & 1 & 0 & 0 & 0 \\
    r_3 & 1 & 1 & 0 & 1 \\
    r_4 & 1 & 0 & 0 & 0 \\
    r_5 & 1 & 1 & 0 & 1 \\
    r_6 & 1 & 0 & 0 & 1 \\
  }  &&
  \mathcal{M'}=\kbordermatrix{%
      & t_1 & t_2 & t_3 & t_4 \\
    r_1 & 1 & 1 & 0 & 1 \\
    r_3 & 1 & 1 & 0 & 1 \\
    r_5 & 1 & 1 & 0 & 1 \\
    \hline
    r_2 & 1 & 0 & 0 & 0 \\
    r_4 & 1 & 0 & 0 & 0 \\
    \hline
    r_6 & 1 & 0 & 0 & 1 \\
  }
\end{align}
 
 As we will see in subsequent chapters, this idea can be utilized to infer biological insights and represent the sequences in a succinct manner. 


%\section{}

 % Introduction

\chapter[\texttt{Quark}]{Clustering of \denovo transcriptome using equivalence classes\footnote{This is a joint work with Avi Srivastava, Laraib Malik and Rob Patro}}

Despite the advanced techniques employed by many modern \denovo transcriptome assemblers, the resulting assemblies often contain a large number of contigs which do not represent full-length transcripts.  These incomplete contigs may result from fractured assemblies, incomplete assemblies due to lack of coverage, errant assembly of chimeric transcripts, or a host of other errors~\cite{transrate}.  Though improved computational methods can reduce the prevalence of such errors, the data itself is often insufficient to guarantee deterministic recovery of all expressed transcripts.  The fractured and incomplete nature of such \denovo assemblies can confound downstream analysis.  For example,~\citet{corset} argue that the large number of contigs that often result from \denovo transcriptome assembly can greatly reduce the statistical power of differential expression analysis.  This results, in part, from the need to correct for testing the many additional hypotheses that arise from considering the large number of assembled contigs (which is likely much greater than the actual number of transcripts expressed in the sample).  Moreover, the sequence-similar contigs generated by \denovo assemblers typically result in a high fraction of ambiguous, multi-mapping reads.  This ambiguity is challenging to resolve, but must be accounted for when performing differential expression analysis at the contig level.

\begin{figure}[!ht]
\includegraphics[width=0.6\textwidth]{Figures/overview}
\centering
\caption{\label{fig:overview}An overview of the \rapclust pipeline. Fragment equivalence classes are computed using \qm, and these classes are used for both contig-level expression quantification and generation of the \ambiggraph. The \ambiggraph is partitioned using \mcl.  The resulting clusters can then be used for downstream analysis (e.g. differential expression).}
\end{figure}

Common pipelines for studying differential gene expression across experimental conditions first align the RNA-seq reads back to the assembled contigs.  Then, they use transcript-level expression estimation tools, such as RSEM~\citep{rsem}, to account for the high degree of multi-mapping ambiguity that results from the substantial sequence similarity between related contigs. To further improve expression estimates, contigs that have high similarity (e.g. that are very sequence-similar or that have many overlapping reads aligning between them) are clustered together as putative transcripts or contigs of the same gene. Statistical methods, such as those discussed in~\cite{kvam, soneson2013comparison}, are then used to identify contigs (or clusters of contigs) that are likely to be differentially expressed across conditions. Clustering of contigs into putative genes can be a crucial step in this analysis, since performing differential expression analysis at the level of clusters reduces the multiple hypothesis testing burden, which can be high in \denovo transcriptomes, owing to the potentially large number of assembled contigs.  Additionally, aggregating contigs into such groups can improve the robustness of expression estimation and hence the accuracy of differential expression analysis.  We note that, if one requires transcript-level differential expression, such a clustering procedure may not always be useful.  However, for many analyses, it is beneficial.

Although clustering may help improve the accuracy of differential expression results, it should be designed to account for the multiple sources of sequence similarity that appear in the assembly.  For example, paralogs should ideally be placed in separate clusters, while isoforms of the same gene should be co-clustered.  The effects of different clustering approaches in the context of analyzing \denovo transcriptomes has previously been explored in depth, e.g. by~\citet{corset}, which largely inspired the current work. Unfortunately, the approaches tend either to have high computational requirements (mainly due to their need to align $10$s or $100$s of millions of reads back to the assembly), or can yield clusters that may poorly reflect the true relationship between contigs and genes~\citep{cdhit}.  In this paper, we present \rapclust, a tool for clustering contigs in \denovo transcriptome assemblies.  \rapclust achieves comparable accuracy to state-of-the-art transcriptome clustering methods, while being much faster. \rapclust works in conjunction with the \sailfish tool, which is already capable of quickly and accurately producing contig-level abundance estimates.  It uses the fragment equivalence classes computed by \sailfish to derive accurate and biologically meaningful clusters at only a marginal extra cost, beyond what is required for quantification.

An overview of the \rapclust pipeline is given in Figure \ref{fig:overview}. \rapclust requires only a transcriptome assembly, the raw sequencing reads, and a description of the experimental design; it yields an accurate contig clustering, along with both contig and cluster-level expression estimates. \rapclust is agnostic to the choice of the underlying \denovo assembler, and can be used with popular tools such as Trinity~\cite{trinity} or Oases~\cite{oases}. \Qm, a recently introduced~\citep{rapmap} and fast alternative to read alignment, is used to map the reads to the assembled contigs. The multi-mapping structure of the sequencing reads with respect to the transcriptome is encoded in the form of {\it fragment equivalence classes}, as discussed in~\ref{subsec:quasimapping_equiv_classes}. These equivalence classes induce a \ambiggraph as detailed in~\ref{subsec:ambig_graph} (the notion of the fragment ambiguity graph has previously proven useful e.g. in the context of re-estimating transcript abundances after updates to an annotation~\citep{reexpress}). After some post-processing, the resulting graph is clustered using \mcl~\citep{mcl}. The computed clusters represent a putative contig-to-gene mapping, which can then be used to aggregate the contig-level expression estimates derived by \sailfish.  These cluster-level expression estimates can then be used for downstream analyses like differential expression testing.


\section{Methods}
\label{sec:methods}

In this work, we make explicit the connection between the successful approach of \denovo transcriptome clustering presented in~\cite{corset}, and the concept of equivalence classes over fragments that has enabled, in part, a new class of ultra-fast methods for transcript-level quantification from RNA-seq data~\cite{sailfish, salmon, kallisto}.  The notion of fragment equivalence classes, as a means of factorizing the likelihood function used in transcript-level quantification, was originally introduced by~\citet{isoem} and~\citet{mmseq} (though a related factorization was used somewhat earlier by~\citet{jiang}). Substantial speed improvements were obtained when traditional alignment of fragments was replaced with much faster procedures like k-mer counting~\citep{sailfish}, lightweight-alignment~\citep{salmon}, pseudoalignment~\citep{kallisto} and \qm~\citep{rapmap}.  All of the ultra-fast transcript quantification tools mentioned above use these fast alternatives to alignment together with the notion of fragment equivalence classes. In our current work, by drawing on this connection and focusing on the rich information exposed by fragment equivalence classes, we frame the transcript clustering problem in the context of the \ambiguitygraph induced by these equivalence classes. This allows for the rapid clustering of contigs, on the basis of both sequence and expression similarity, using only small intermediate space.

\subsection{Computing Equivalence Classes via Quasi-mapping}
\label{subsec:quasimapping_equiv_classes}

The concept of \qm, which provides information about the transcripts, positions and orientations from which a fragment has possibly originated, but not the base-to-base alignment by which the fragment corresponds to the transcript, has recently been introduced in~\citet{rapmap}. There, it is suggested that \qm may be adopted as a much-faster alternative to fragment alignment when the base-to-base alignments are not required for the task being performed. \citeauthor{rapmap} describe an efficient implementation \qm in the tool \rapmap, and demonstrate how integrating \qm in the \sailfish software for transcript-level quantification led to considerable improvements in accuracy and speed. Here, we rely on \qm to allow for very fast and accurate determination of fragment equivalence classes, which is crucial to the approach we adopt below. We note that, though \qm does not compute a nucleotide-level alignment, it is sensitive to even small differences in related reference sequences.  Thus, it can accurately map fragments to e.g. the appropriate paralog, even if the fragment contains only a single SNP differentiating the alternative reference sequences.  We refer the reader to~\citep{rapmap} for details of the \qm concept and the particular algorithm used by \rapmap.

% Describe the equivalence classes
We use the same definition of equivalence classes defined in \ref{subsec:gen_equiv_classes} but in the context of fragments instead of reads.
Just as the case of reads we define an equivalence relation over fragments, based on the set of transcripts to which they map.  The set of fragments related under this definition constitutes a fragment equivalence class.  Let $\mathcal{M}\left(f_i\right)$ be the set of transcripts to which fragment $f_i$ maps, and let $\mathcal{M}\left(f_j\right)$ be the set of transcripts to which fragment $f_j$ maps.  We say that $f_i \sim f_j$ if and only if $\mathcal{M}\left(f_i\right) = \mathcal{M}\left(f_j\right)$. Consequently, a fragment equivalence class is a set of fragments such that, for every pair $f_i$ and $f_j$ in the class, $f_i \sim f_j$. An equivalence class can be uniquely labeled based on the set of transcripts to which the fragments contained in this class map.  We define the label of equivalence class $\eqclass{f_i} = \{ f_j \in \mathcal{F} \mid f_j \sim f_i\}$ as $\eqlabel{\eqclass{f_i}}$.  It is important to remember that, though the label consists of transcript names, the equivalence relation itself is defined over sequenced fragments and not transcripts.  Finally, in addition to a label, we denote the count of each equivalence class $C_i$ by $\eqcount{C_i}$; this is simply the number of equivalent fragments in $C_i$.

\subsection{Quantification and Graph Determination}
\label{subsec:ambig_graph}

The fragment equivalence classes, as described above, are already computed internally by \\ \sailfish \citep{sailfish}. We have modified \sailfish to write these equivalence classes to a file once quantification is complete (this behavior is enabled with the \texttt{--dumpEq} flag).  This yields, for each \textit{sample}, a collection of equivalence classes, along with their associated labels and counts. To construct the complete \ambiggraph of the \textit{experiment} we need to aggregate these equivalence classes over all of the processed samples. In fact, this aggregation is relatively simple since the labels of fragment equivalence classes are deterministic and stable across samples (i.e. they depend only on the underlying transcriptome). We generate a single collection $\mathcal{C}$ of equivalence classes by merging the classes $\mathcal{C}_1, \dots, \mathcal{C}_M$, from all samples, where $M$ is the number of samples. Here, $\mathcal{C}$ contains the union of equivalence classes from $\mathcal{C}_1, \dots, \mathcal{C}_M$, and classes that appear in more than one sample of $\mathcal{C}_1, \dots, \mathcal{C}_M$ simply have their respective read count summed.  The time and space requirement for this aggregation algorithm is linear in the size of input.

% Now, define the graph in terms of the classes
For a given experiment, the collection $\mathcal{C} = \{C_1,C_2,\ldots,C_k\}$ of equivalence classes induces a weighted, undirected, \ambiggraph $G = \left(V, E\right)$.  Here, $V = T$, where $T$ is the set of transcripts in the original transcriptome --- and $E = \{ \{t_i, t_j\} \mid \exists\; C_\ell \in \mathcal{C} \text{ where } \{t_i, t_j\} \subseteq \eqlabel{C_\ell}\}$ --- that is $t_i$ and $t_j$ are connected by an edge if and only if they both appear in the label of at least one equivalence class.  For a given edge $\{t_i, t_j\}$, its weight is given by $w\left(t_i, t_j\right) = \nicefrac{N_{ij}}{\min\left(N_i, N_j\right)}$, where
    %
    \[
    N_{ij} = 
    \sum_{\substack{C_\ell \in \mathcal{C} \mid \\ \{t_i, t_j\} \subseteq \eqlabel{C_\ell}}} \eqcount{C_\ell}
        ,\, N_i = \sum_{\substack{C_\ell \in \mathcal{C} \mid \\ t_i \in \eqlabel{C_\ell}}} \eqcount{C_\ell} \text{ and }
    N_j = \sum_{\substack{C_\ell \in \mathcal{C} \mid \\ t_j \in \eqlabel{C_\ell}}} \eqcount{C_\ell}
    \]

While the samples we process from a \denovo RNA-seq experiment may contain $10$s to $100$s of millions of fragments, the number of nodes in $G$ is determined by the number of contigs.  Further, the number of edges is bounded by the \textit{complexity} of the transcriptome (i.e. the degree of alternative splicing and paralogy in the underlying transcriptome), and, therefore, mostly independent of the number of fragments processed. 

For the transcriptomes and samples we analyze in this paper, the number of equivalence classes never rises above a few hundred thousand, and is typically orders of magnitude smaller than the number of fragments (see~\ref{tab:data} for the number of equivalence classes in the different data sets).  Thus, if the equivalence classes can be computed efficiently from the transcriptome and sequencing data, then the \ambiggraph can be constructed efficiently in terms of time and space.

\subsection{Processing and Partitioning the Mapping Ambiguity Graph}
\label{sec:mag_filter}

Once the \ambiggraph $G$ is constructed, it is filtered, as described below, to yield a graph $G'$.  $G'$ is then clustered using an off-the-shelf graph clustering algorithm.  Currently, \rapclust employs two simple filters. The first filter removes nodes from $G$ that have fewer than some nominal threshold of read support over all samples in an experiment. We adopt the cutoff used by~\citep{corset}, and remove any contig with $10$ or fewer mapped reads from the \ambiggraph.  

The second filter, also inspired from~\citep{corset}, is used to remove edges between pairs of contigs that are likely to arise from paralogous genes.  Specifically, this filter tests the hypothesis that the constant of proportionality between the number of reads mapping to $t_i$ and $t_j$ does not vary (by a statistically significant amount) across conditions. 

This is done by testing the hypothesis ($H_0$) that the constant of proportionality remains constant across conditions versus the hypothesis  ($H_1$) that it does not. The log-likelihoods of the competing hypotheses $H_0$ and $H_1$ are computed according to ~\eref{eqn:likelihood_null} and \eref{eqn:likelihood_alt} respectively. A likelihood ratio test is performed, and edges $\{t_i, t_j\}$ from $G$ where $2 \left(\ell_1 - \ell_0\right) > 20$ are removed (we refer the reader to~\citep{corset} for a justification of the cutoff used in the likelihood ratio test).  This test, of course, makes some simplifying assumptions, since isoforms of the same gene might exhibit behavior consistent with $H_1$ (e.g. if isoform switching occurs between conditions).  However, we found that this filter correctly separates transcripts from paralogous genes more often than it incorrectly separates transcripts of the same gene.  Thus, applying this filter leads to a slight increase in \rapclust's precision and a typically smaller decrease in its recall.

\begin{equation}
\ell_0 = \sum_{c} \left[\left(X_i^{c} \cdot \log\left(r_{ij} \mu_j^{c}\right)\right) - \left(r_{ij} \mu_j^{c}\right)\right] + \left[\left(X_i^{c} \cdot \log\left(\mu_j^{c}\right)\right) - \left(\mu_j^{c}\right)\right]
\label{eqn:likelihood_null}
\end{equation}
\begin{equation}
\ell_1 = \sum_{c} \left[\left(X_i^{c} \cdot \log\left(r_{ij}^{c} X_j^{c}\right)\right) - \left(r_{ij}^{c} X_j^{c}\right)\right] + \left[\left(X_i^{c} \cdot \log\left(X_j^{c}\right)\right) - \left(X_j^{c}\right)\right]
\label{eqn:likelihood_alt}
\end{equation}

where

$$r_{ij}^{c} = \frac{X_i^{c}}{X_j^{c}},\; r_{ij} = \frac{\sum_{c}X_i^{c}}{\sum_{c}X_j^{c}},\; \text{ and } \mu_{j}^{c} = \frac{X_{i}^{c} + X_{j}^{c}}{1 + r_{ij}},$$

and $X_i^{c}$ denotes the number of reads mapping to contig $i$ under the $j^{\text{th}}$ condition (summed over all replicates of a condition for simplicity).

After applying both of these filters in sequence, we obtain the final processed graph $G'$, which is then clustered using \mcl~\cite{mcl}.  For the sake of simplicity, and to avoid an unnecessary dependence on parameters, we generated all the clusterings in this paper using \mcl's default parameters, and applying no additional cutoff or modification to the edge weights. 


\section{Results}
To analyze the performance of \rapclust, we have benchmarked its running time, space usage, and accuracy against \corset~\citep{corset} and \cdhit~\citep{cdhit,cdhit2} (here, we consider \cdhit\texttt{-EST}). Note that \cdhit does not provide any quantification results.  Therefore, in~\sref{subsec:DGE}, we considered the clustering computed by \cdhit, but estimated the expression of those clusters by aggregating the contig-level expression estimates computed by \sailfish.  Testing was performed on 3 datasets, human primary lung fibroblast samples, with and without a small interfering RNA (siRNA) knock down of HOXA1~\citep{humandata} (Gene Expression Omnibus accession GSE37704), yeast grown under batch and chemostat conditions~\citep{yeastdata} (Sequence Read Archive (SRA) accessions SRR453566 to SRR453571) and male and female chicken embryonic tissue~\citep{chickendata} (SRA accession SRA055442). For each of these data sets, we performed clustering of Trinity~\citep{trinity} \denovo assemblies, which were obtained from~\citep{corset_data}.

All experiments were performed on a 64-bit Linux server, running Ubuntu 14.04, with 4 hexacore Intel Xeon E5-4607 v2 CPUs (with hyper-threading) running at 2.60GHz and 256GB of RAM. Wall-clock time was recorded using the Unix \texttt{time} command. \tref{tab:data} gives a brief description of the input data.

\begin{table}[ht!]
\centering
\caption{\label{tab:data}Summary statistics for the transcriptomes and experimental samples on which the experiments were carried out, as well as the average number of fragment equivalence classes and the size of the resulting \ambiggraph.}
\begin{tabular}{lrrr}
\toprule
{} & Yeast & Human & Chicken \\
\midrule
\# contigs        &  \num{7353}  &  \num{107389} & \num{335377}  \\
\# samples        &  6  &  6 & 8  \\
Total (paired-end) reads       &  $\sim$\num{36000000}  & $\sim$\num{116000000} & $\sim$\num{181402780}  \\
Avg \# eq. classes (across samples) & \num{5197}            & \num{100535}          & \num{222216} \\
\# edges in \ambiggraph &    \num{6195}                     & \num{212481}         & \num{2063524} \\
\bottomrule
\end{tabular}
\end{table}

\subsection{Time and Space Requirements}

Here, we report, for each clustering method, the time required to perform the clustering as well as the total size of the intermediate and result files written to disk.  It is important to note that, unlike \rapclust and \corset, \cdhit neither counts reads nor performs quantification. In order to use the clusters resulting from \cdhit in a typical analysis (e.g. quantification and differential expression testing), one would need to either align reads to the \cdhit clusters, or perform quantification on these clusters using the sequenced reads, both of which would add to the time and disk space required.

\begin{table}
\centering
\caption{\label{tab:time_space}\rapclust is \textit{substantially} faster than \corset, and requires only a small fraction of the intermediate disk space used by \corset.  The majority of space savings for \rapclust come as a result of avoiding alignment and generation of the intermediate \texttt{BAM} files. In addition to that, the percentage of reads mapped using \rapclust is much higher. Originally, the percentage of mapped reads for \rapclust was even higher than what is reported here, but we subsequently modified our default behavior to discard orphan mappings to be consistent with the behavior of the alignments provided to \corset.} %Discarding a smaller amount of data, hence, results in a greater number of clusters.}
\hskip-0.35in
\begin{tabulary}{1cm}{lrr|rr|rr}
\toprule
{} & \multicolumn{2}{c}{Yeast} & \multicolumn{2}{c}{Human} & \multicolumn{2}{c}{Chicken} \\
\midrule
{} &  \rapclust & \corset & \rapclust & \corset & \rapclust & \corset \\
\midrule
Time(min)        &  5.12  & 37.25 & 22.67  &211.67 & 64.18  &  453 \\
Space(Gb)        &  0.005  &  5.7   & 0.092  &  22   & 0.49    &  145   \\
$\%$ of reads    &  88.17    &  62.32   & 93.04    & 77.94   &  88.80   &  60.99\\
%Clusters        &  4071  &  4145 & xx   &  52575 & 69107 & xx   &  204189 & 181333 & xx\\
\bottomrule
\end{tabulary}
\end{table}

For each transcriptome, the time reported for \corset is the sum of the time required to trim the reads using \trimmomatic~\citep{trimmomatic}, align the reads using \Bowtie, and cluster the contigs using the resulting \texttt{BAM} files (this adopts the protocol and parameters suggested in the \corset documentation).  For \rapclust, the time reported is the time required to run \sailfish (v0.9.1) on all the samples, plus the time required to generate, filter and cluster the \ambiggraph.  \sailfish is run with the \texttt{--dumpEq} option to write to disk the equivalence classes computed during the quantification of each sample.  \trimmomatic, \Bowtie, and \sailfish were each run with $4$ threads.  For \cdhit, only the time required to run \cdhit is reported.  To determine the disk space required for analysis of each transcriptome using \corset, we sum the size of the \texttt{BAM} files produced by \Bowtie and the ``count'' and ``cluster'' files produced by \corset.  For \rapclust, we determined the required intermediate disk space by summing the sizes of the \sailfish quantification directories and the ``graph'' and ``cluster'' files.  Since both methods require the same input in terms of the assembled transcriptome and set of reads, we don't count these toward the space requirements.  The complete timing results (from assembly and raw reads to computed clusters) for \rapclust and \corset are presented in~\tref{tab:time_space}.  The times required for \textit{just} the clustering steps of \rapclust and \corset, as well as the time required by \cdhit, are reported in~\tref{tab:time_space_2}.  We choose to report these results separately to highlight the fact that, since \cdhit only performs clustering, if one wishes to carry out expression analysis on the clusters computed by \cdhit, she would additionally have to either align the sequenced fragments or perform contig-level abundance estimation on the samples, which could take much longer.

\begin{table}
\centering
\caption{\label{tab:time_space_2}The overall time taken for \corset and \rapclust are dominated by the time taken for alignment and quantification respectively. However, when we consider just the time required for clustering (i.e. after alignments have been generated for \corset and after the fragment equivalence classes have been generated for \rapclust), we observe that \rapclust and \cdhit are considerably faster than \corset.  As \corset's clustering phase is single-threaded, we provide times for all methods in this table using only a single thread. (RC = \rapclust, CD = \cdhit, CT = \corset)}
\begin{tabulary}{1cm}{lrrr|rrr|rrr}
\toprule
{} & \multicolumn{3}{c}{Yeast} & \multicolumn{3}{c}{Human} & \multicolumn{3}{c}{Chicken} \\
\midrule
{} &  RC & CD & CT &  RC & CD & CT & RC & CD & CT \\
\midrule

Time(min)        &  0.04  & 0.2  & 2.8 &  0.82 & 4.02 & 16.25 &   5.29  &  36.5 & 87\\
\bottomrule
\end{tabulary}
\end{table}

\subsection{Assessing Cluster Quality}
\label{subsec:quality}
\begin{figure}[htb]
    \centering
    \subcaptionbox{Accuracy for yeast\label{fig:concordance_yeast}}[0.3\textwidth]{
        \includegraphics[width=0.3\textwidth]{Figures/concordance_yeast_legend}}
    \subcaptionbox{Accuracy for human\label{fig:concordance_human}}[0.3\textwidth]{
        \includegraphics[width=0.3\textwidth]{Figures/concordance_human}}
    \subcaptionbox{Accuracy for chicken\label{fig:concordance_chicken}}[0.3\textwidth]{
        \includegraphics[width=0.3\textwidth]{Figures/concordance_chicken}}
    \caption{The precision, recall, and F1-score of the \rapclust, \corset and \cdhit based clusterings, with respect to ground-truth annotations, on the yeast a, human b and chicken c data sets.\label{fig:cluster_quality}}
\end{figure}

We assessed the quality of the clusters obtained by the various tools using two different metrics. The ground truth labels for \denovo assembled contigs were taken from~\citep{corset_data}, and the process of obtaining these labels is described in~\citep{corset}.  It is important to note that not all contigs can be labeled with an annotated gene, and unlabeled contigs were omitted when computing the metrics below.


First, we considered the precision and recall metrics used by~\citet{corset}.  Here, pairs of contigs are classified based on whether their cluster labels match their annotated gene labels.  Specifically, a pair of contigs labeled with the same gene is considered a true-positive (TP) if the pair is co-clustered and a false-negative (FN) if the contigs are placed in separate clusters.  Likewise, if a pair of contigs labeled with different genes is placed into the same cluster, it is considered a false-positive (FP) and if they are placed in different clusters, it is considered a true-negative (TN).  From these pairwise counts, the precision and recall can be computed as $\text{Precision} = \nicefrac{\text{TP}}{\text{TP}+\text{FP}}$ and $\text{Recall} = \nicefrac{\text{TP}}{\text{TP}+\text{FN}}$.  A higher precision signifies that, when contigs are co-clustered, they are more likely to have originated from the same underlying gene.  A higher recall, on the other hand, suggests that more contigs originating from the same gene tend to be co-clustered.  Typically, precision and recall are competing objectives, as \textit{over clustering} will improve recall but harm precision while \textit{under clustering} will improve precision but harm recall.  Commonly, the $\text{F1-Score} = 2 \left(\nicefrac{\text{Precision} \cdot \text{Recall}}{\text{Precision} + \text{Recall}}\right)$ is used as a single metric to summarize performance in terms of both the precision and recall. \fref{fig:cluster_quality} shows the accuracy of all three tools on the three assemblies in terms of Precision, Recall and F1-Score.  \corset and \rapclust generate similar clusters, with \rapclust generally yielding slightly higher recall than \corset.  \cdhit, however, tends to do a much poorer job at trading off between these competing objectives.  It provides similar (or, in case of the yeast data, higher) precision to \corset and \rapclust, but the resulting clusters exhibit much lower recall.

Second, we considered how similar the clusters obtained using the different methods are to the ground truth clustering, which groups together all contigs labeled with the same gene.  To assess this similarity, we used the variation of information (VI)~\citep{vi}. The VI is defined over a pair of clusterings, and quantifies the information lost and gained when moving from one clustering to the other.  It allows one to measure how similar two clusterings are, regardless of the specific names or labels chosen for the clusters.  The lower the VI between a pair of clusters, the more similar they are.  To compute the VI between the true clustering $C_T$ and that obtained by a particular method $C_M$, we discarded all contigs in $C_M$ that are not labeled with a gene name, while retaining the clustering relations among the remaining contigs.  If any contigs that correspond to annotated genes do not exist in $C_M$ (since they may be discarded, e.g., by the read count filter described in~\sref{sec:mag_filter}), we considered them to come from a single cluster, which is given a new label.  We called the resulting clustering $C_{M'}$. Hence, $C_T$ and $C_{M'}$ are defined over the same set of contigs, and the similarity between them can be computed directly using the VI.  The VI results are presented in~\Cref{tab:VI}.  \rapclust and \corset seem to yield similar results, with \rapclust obtaining a slightly lower (better) VI. However, we observed a \textit{marked} difference between these two and \cdhit, whose clusters exhibit a substantially larger VI, especially on the human and chicken data.

\begin{table}
\caption{\label{tab:VI}The variation of information between contig to gene mapping using genome-based mapping approach and the clusters generated using \rapclust, \corset and \cdhit.  For each assembly, the clustering producing the lowest variation of information with respect to the true clustering is set in bold; \rapclust achieves the lowest VI on all assemblies.}
\centering
\begin{tabular}{cccc}
  \toprule
  & \rapclust & \corset & \cdhit \\
  \midrule
  Chicken & \textbf{0.127} & 0.191 & 2.01 \\
  Human & \textbf{0.712} & 0.735 & 1.24 \\
  Yeast & \textbf{0.176} & 0.178 & 0.216 \\
  %Chicken & \cellcolor{green}0.145 & \cellcolor{yellow}0.191 & 0.0 \\
  %\hline
  %Human & \cellcolor{green}0.712 & \cellcolor{yellow}0.735 & \cellcolor{red}0.881 \\ 
  %\hline
  %Yeast & \cellcolor{green}0.176 & \cellcolor{yellow}0.178 & \cellcolor{red}0.2 \\
  \bottomrule
\end{tabular}
\end{table}

\subsection{Differential Gene Expression}
\label{subsec:DGE}
We also tested the ability to recover gene-level differential expression using the clusterings produced by the different methods. \Denovo transcriptome assemblers have a tendency to produce many (often fractured) contigs. This tends to confound downstream differential expression analyses, due, in part, to the difficulty of quantifying fractured or incorrectly assembled contigs, and due, in part, to the potentially large number of extra statistical tests being performed (that must be corrected for). By estimating expression, and performing differential expression testing at the cluster level, one might expect to simultaneously reduce the multiple hypothesis testing burden and ``average out'' some of the mistakes made in contig-level abundance estimation.

For gene-level differential expression analysis, we compared the genes called as differentially expressed under each of the different clusterings versus the genes detected as differentially expressed under the true contig-to-gene mapping (again, note that not all contigs are annotated with a gene label). We generated ``ground truth'' gene-level abundance estimates based on the contig-level abundance estimates computed by \sailfish, and the true contig-to-gene mapping.  Using the tximport~\citep{tximport} R package, we loaded the \sailfish expression estimates, aggregated them to the gene level, and prepared them for use with DEseq2~\citep{deseq2}.  A 2-condition differential expression test was performed in each data set; for human this was in between the conditions with and without the HOX1A knockdown, for yeast it was in the batch and chemostat growth conditions, and for chicken it was in the male and female samples (we collapsed the different tissues within each sex). We then obtained corrected p-values for the hypothesis that each gene is differentially expressed across the conditions we considered. We considered genes having a corrected p-value less than or equal to $0.05$ as differentially expressed. We estimated differential expression under the \rapclust and \cdhit clusterings in the same manner, where the quantification estimates were held fixed, but the true contig-to-gene mapping was replaced with the contig-to-cluster mapping produced by these tools.  For \corset, a count matrix is directly provided that was used as input to DESeq2.

As a metric of comparison, we examined the rate at which true positive differentially expressed genes were recovered versus the rate at which false positive differentially expressed genes were called. Each predicted cluster was labeled with the union of all of the genes with which its constituent contigs were labeled. We sorted the list of clusters by p-value, and processed them in the following manner: when we encountered a cluster, we intersected its set of labeled genes with the set of truly differentially expressed genes. Any genes that had not already been encountered in a more highly-ranked cluster were considered as true positive predictions. Likewise, any genes that appeared in the cluster, but which did not occur in the true set of differentially expressed genes were considered as false positives (if the genes had not already been encountered in a more highly-ranked cluster). We stopped processing the clusters once their adjusted p-value exceeded $0.05$ (since, under common threshold, such clusters would likely not be considered to be differentially expressed). For the true and false positive predictions we encountered, the associated negative p-value of the associated cluster was used as the corresponding ``score''. The ROC curves were generated using the CROC~\cite{croc} Python package.

\begin{figure}[thb!]
    \centering
    \subcaptionbox*{Yeast\label{fig:dge_yeast}}[0.3\textwidth]{
        \includegraphics[width=0.3\textwidth]{Figures/yeast_dge_curves}}
    \subcaptionbox*{Human\label{fig:dge_human}}[0.3\textwidth]{
        \includegraphics[width=0.3\textwidth]{Figures/human_dge_curves}}
    \subcaptionbox*{Chicken\label{fig:dge_chicken}}[0.3\textwidth]{
        \includegraphics[width=0.3\textwidth]{Figures/chicken_dge_curves}}
    \caption{ROC curves showing the recovery rate of \rapclust, \corset, and \cdhit's clusters in recovering 
        differentially expressed genes in each data set.\label{fig:ROC}}
\end{figure}

Overall, poor precision or recall in terms of clustering may lead to detection of fewer truly differentially expressed genes, spurious identification of differential expression, or weak statistical evidence for differential expression.~\fref{fig:ROC} shows that the rate at which \rapclust recovers true positives versus false positives is higher than that of \cdhit in $2$ of the $3$ assemblies, and is higher than that of \corset in all the assemblies, as represented by the respective area under the curves. The benefit of \rapclust is particularly apparent in the chicken assembly. Since the quantification results produced by \sailfish tend to be reasonable, even when the clustering is fairly poor, this may explain the relatively good performance of the \cdhit clustering in yeast (relative to the rather poor quality of the clustering, as evaluated in~\Cref{subsec:quality}).

 % Background Theory 

\chapter{Compression on the basis of equivalence classes}

\section{Related Compression Techniques}
Compression of sequencing short reads becomes crucial with the lowering cost of sequencing technology. The rapid technological development enables the generation of petabytes of data on servers worldwide. Apart from size, often succinct representation \citep{Pritt2016} also yields almost accurate results with a much smaller memory footprint. Nucleotide sequence compression can largely be divided into two paradigms, one is reference based compression where the reference is needed to transfer to the decoder end along with the compressed data ~\citep{Canovas2014}, \citep{Fritz2011}, \citep{Li2014}. Another is reference free compression, \citep{adjeroh2002dna},\citep{Bonfield_2014}, \citep{Hach2012} where the read sequences are compressed independent of reference sequence, barring the burden of transferring reference. Such compressions are commonly known as \Denovo compression. A robust and widely used \denovo compressor, LEON \citep{Benoit2015}, constructs a de Bruijn graph from the k-mer counts table extracted build on the reads. Reads are then mapped to the newly constructed de Bruijn graph, and are stored in form of anchor address, read size and bifurcation list.   

Reference based compressions start with the BAM file produced by state of the art aligners such as \Bowtie. Typically these algorithms store the edits after aligning the sequences to reference. \citet{Fritz2011} is one such widely used tool A serious bottleneck of such approaches is to store all meta information in the BAM file which can be regenerated re-aligning the reads to the provided reference sequence. This problem is partially addressed by fastq \citep{bonfield2013compression}, where a new alignment technique is being implemented to navigate through the problem of storing meta-data footprint. 



There are few mappers which does compression of genome/transcriptome for the better mapping, and also compress reads on the way. Recently published CORA ~\citep{Yorukoglu2016} is a compressive read mapper. The working principle of CORA is interesting and worth mentioning as concept of equivalence classes is also used there, but carries a different meaning. CORA converts the fastq reads to non redundent k-mer sets. Furthermore, CORA does a redundancy removal on reference genome, by self mapping. The positions where a k-mer maps form an equivalence class. Two equivalence classes are regarded as concordant if all positions of one equivalence class are in one nucleotide shift distance from another equivalence class. 

Apart from reference based and \denovo mappers, few mappers lie in the middle, which  stores a part of the reference that is shared between reads. Kpath \citep{Kingsford2015} is a tool that stores the shared reference as that acts as a backbone for a statistical generative model. An arithmetic coding is performed on the reads.


 % Experimental Setup

\chapter{Discussion and Future Work}
\label{sec:conclusion}

In this report we broadly outlined the concept of \eq in the context of RNA-seq experiments, and have shown two useful applications: \rapclust and \quark. 
\rapclust implements a fast and accurate methodology for the data-driven clustering of \denovo transcriptome assemblies. By making explicit the connection of the method of \corset~\citep{corset} and the fragment equivalence classes that have recently been proven useful in the development of accurate and fast RNA-seq quantification tools~\cite{sailfish, salmon, kallisto}, we have demonstrated how state-of-the-art transcript clustering results can be obtained much more quickly than is possible with existing tools. Furthermore, when working directly in the compact and efficient representation of fragment equivalence classes, this clustering requires only a marginal computational and storage cost beyond what is already required for \qm-based transcript-level quantification. 




\section{Possible extension of \rapclust}

There are many interesting directions for future work on this problem. Here, we mention only the most obvious. First, we believe that the quality of the resulting clusters could be improved through a data-driven selection of the appropriate cutoff parameters.  Currently, we have directly adopted the parameters suggested in~\citep{corset}, which are selected partly independent of the underlying data.  Preliminary experiments have suggested that one might be able to obtain substantially better clusters by selecting the transcript count cutoff, an edge-weight cutoff, and the log fold-change likelihood cutoff in a manner that is more data-adaptive.  However, automatically selecting these cutoffs in a general, yet rigorous fashion is a topic for future work.  Another potential improvement on the current methodology would be to adopt a more robust log fold-change test, that may be more accurate in separating contigs that do not originate from the same gene. \sailfish is capable of producing not only transcript-level abundances, but also estimates of the variance of each predicted abundance via posterior Gibbs sampling or bootstraps.  This variance information can be incorporated into the estimates of log fold-change differences to allow for increased precision in separating potential paralogs. While the existing method works well in the completely \denovo context (i.e. even when the genomes or transcriptomes of closely related organisms may not be available), integrating homology information, when available, has the potential to improve the clustering results (and provide meaningful biological annotations for the clusters). The best way to integrate this information is an exciting direction for future work. Finally, we believe that sequence-level comparison and analysis of the resulting clusters of contigs could reveal important information about the nature of the transcripts present in the samples.  For example, one could imagine ``reverse-engineering'' the splicing patterns present in the transcripts occurring in the same cluster.  This would allow one to build a virtual gene model, even in the completely \denovo context, which could then be used downstream, such as for differential splicing analyses. 


The concept of {\it equivalence classes} is generic and can be extended in multiple ways, such as identification of novel transcripts, discovering splice sites, improving quality of \denovo assembly etc. In case of annotated reference, \eq represents possible isoforms, whereas in \denovo world the meaning of \eq is closely related to sequences that potentially share a strong sequence similarity, given the read data. This connection can be used to identify novel transcripts while performing the mapping and quantification process. It would be interesting to explore the correlation between change of structure of \eq in the framework of control-treatment experiment. The \denovo assembly process can be substantially improved by collapsing \eq from multiple samples and conditions. 

%We use an expectation-maximization (EM) algorithm to manipulate edge weights in the graph based on the relationship within contigs from the annotated species and the sequence similarity between contigs from the two organisms (explained in more detail in the Online Methods section). Annotations are mapped onto the graph using BLAST between the sequences. The EM algorithm then assigns annotations to unlabeled contigs using the graph-based semi-supervised learning algorithm, Modified Adsorption (MAD, implemented in publically available software - add an explanation of why we use this method). . In order to further improve contig annotation, edge weights are changed (explained in detail in Online Methods) and the label propogation algorithm re-run. This is repeated until the number of edges in the graph converges. We show that this method significantly increases the number of annotated contigs as compared to a simple BLAST (add results showing numbers, add a sentence about result) and also improves the quality of clustering contigs. 
 % Experiment 1

%\input{Chapters/Chapter5} % Experiment 2

%\input{Chapters/Chapter6} % Results and Discussion

%\input{Chapters/Chapter7} % Conclusion

%% ----------------------------------------------------------------
% Now begin the Appendices, including them as separate files

\addtocontents{toc}{\vspace{2em}} % Add a gap in the Contents, for aesthetics

%\appendix % Cue to tell LaTeX that the following 'chapters' are Appendices

%\input{Appendices/AppendixA}	% Appendix Title

%\input{Appendices/AppendixB} % Appendix Title

%\input{Appendices/AppendixC} % Appendix Title

\addtocontents{toc}{\vspace{2em}}  % Add a gap in the Contents, for aesthetics
\backmatter

%% ----------------------------------------------------------------
\label{Bibliography}
\lhead{\emph{Bibliography}}  % Change the left side page header to "Bibliography"
\bibliographystyle{unsrtnat}  % Use the "unsrtnat" BibTeX style for formatting the Bibliography
\bibliography{Bibliography}  % The references (bibliography) information are stored in the file named "Bibliography.bib"

\end{document}  % The End
%% ----------------------------------------------------------------